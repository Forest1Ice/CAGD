\documentclass[utf8]{ctexart}

\usepackage[a4paper,left=1.25in,right=1.25in,top=1in,bottom=1in]{geometry}
\usepackage{listings}
\usepackage{graphicx}
\usepackage{caption}
\usepackage{subfigure}
\usepackage{booktabs}
\usepackage{amsmath}
\usepackage{amsthm}
\usepackage{amsfonts}
\usepackage{float}
\usepackage{indentfirst}
\usepackage{tikz}
\usetikzlibrary{shapes,arrows}
\usetikzlibrary{shapes.geometric, arrows}
\usepackage{algorithm}
\usepackage{algorithmic}
\usepackage{newclude}
\usepackage[perpage]{footmisc}

\graphicspath{ {images/} }
\raggedbottom	% 令页面在垂直方向向顶部对齐
\renewcommand\qedsymbol{QED}
\newcommand{\sign}[1]{\mathrm{sgn}(#1)}
\everymath{\displaystyle}   % 行内公式采用行间公式格式排列
\pagestyle{plain}

\title{《计算机辅助几何设计》第十次作业}
\author{姓名:殷文良\qquad 学号:12435063}
\date{\today}
\ctexset { section = { format={\Large \bfseries } } }

\begin{document}
\maketitle
\section*{思考题 1}
\subsection*{1.}
\begin{proof}
    设存在系数$a_{ij}$使得
    $$
    \sum_{i+j\leq n}a_{ij}J_{i,j,k}^n(u,v,w) = \sum_{i+j\leq n}a_{ij}\frac{n!}{i!j!k!}u^iv^jw^k = 0.
    $$
    上式两端同除以$w^n$,可得
    $$
    \sum_{i+j\leq n}a_{ij}\frac{n!}{i!j!k!}\frac{u^iv^j}{w^{n-k}} = \sum_{i+j\leq n}a_{ij}\frac{n!}{i!j!k!}\frac{u^i}{w^{i}}\frac{v^j}{w^j} := \sum_{i+j\leq n}a_{ij}\frac{n!}{i!j!k!}x^iy^j,
    $$
    其中,$x := \frac{u}{w}, y := \frac{v}{w}$。\\
    由二元幂函数的线性无关性,可知$\forall (i,j,k), a_{ij}\frac{n!}{i!j!k!}=0$,即$a_{ij} = 0$,从而三角形上定义的
    Bernstein函数是线性无关的。
\end{proof}

\subsection*{2.}
\begin{proof}
    由于$\alpha,\beta,\gamma=n-\alpha-\beta>0$,根据三角域 Bernstein 基函数的权性,可得
    $$
    \sum_{i+j+k=n}\frac{(n-\alpha-\beta)!}{(i-\alpha)!(j-\beta)!k!}u^{i-\alpha}v^{j-\beta}w^k = 1.
    $$
    上式两端同乘以$u^{\alpha}v^{\beta}$,可得
    $$
    \begin{aligned}
        u^{\alpha}v^{\beta} &= \sum_{i+j+k=n}\frac{(n-\alpha-\beta)!}{(i-\alpha)!(j-\beta)!k!}u^{i}v^{j}w^k\\
        &= \frac{\alpha!\beta!\gamma!}{n!}\sum_{i+j+k=n}\frac{n!}{[(i-\alpha)!\alpha!][(j-\beta)!\beta!]k!}u^{i}v^{j}w^k\\
        &= \frac{\alpha!\beta!\gamma!}{n!}\sum_{i+j+k=n}\frac{i!j!}{[(i-\alpha)!\alpha!][(j-\beta)!\beta!]}\frac{n!}{i!j!k!}u^{i}v^{j}w^k\\
        &= \frac{\alpha!\beta!\gamma!}{n!}\sum_{i+j+k=n}\binom{i}{\alpha}\binom{j}{\beta}B_{ijk}^n(u,v,w).
    \end{aligned}
    $$
\end{proof}

\subsection*{3.}
\begin{proof}
    由第2题的结论,可得
    $$
    x^iy^j = \frac{i!j!(n-i-j)!}{n!}\sum_{r+s+t=n}\binom{r}{i}\binom{s}{j}B_{rst}^n(x,y,z),
    $$
    其中,$z=1-x-y$。因此,二元多项式可转化为如下的Bezier三角片形式
    $$
    \begin{aligned}
    b(x,y) &= \sum_{0\leq i+j\leq n}a_{ij}x^iy^j\\
    &= \sum_{0\leq i+j\leq n}a_{ij}\frac{i!j!(n-i-j)!}{n!}\sum_{r+s+t=n}\binom{r}{i}\binom{s}{j}B_{rst}^n(x,y,z).
    \end{aligned}
    $$
\end{proof}

\section*{4.}
\begin{proof}
    对于给定的$n$次Bézier三角片$B^n(P) = \sum_{i+j+k=n}Q_{i,j,k}J_{i,j,k}^n(P)$,\newline
    取三角形$T = Q_{i+1,j,k},Q_{i,j+1,k},Q_{i,j,k+1} := Q_0,Q_1,Q_2(i+j+k=n-1)$。\\
    从$Q_1$处出发向对边$Q_0Q_2$连一直线交于点$Q_3$,设$|Q_0Q_3| = s|Q_0Q_2|, 0\leq s\leq 1$。则对于$Q_1Q_3$上任意一点$Q$,根据几何性质有
    $$
    Q = (1-s)tQ_0 + (1 - t)Q_1 + stQ_2,\quad 0\leq t \leq 1.
    $$
    于是,Bézier三角片中参数定义在该直线上的曲线方程为
    $$
    \begin{aligned}
    c(t) = B^n(Q) &= \sum_{i+j+k=n}Q_{i,j,k}J_{i,j,k}^n(Q)
        &= \sum_{i+j+k=n}Q_{i,j,k}J_{i,j,k}^n((1-s)tQ_0 + (1 - t)Q_1 + stQ_2),\\
        0 \leq t \leq 1.
    \end{aligned}
        $$
\end{proof}

\end{document}
