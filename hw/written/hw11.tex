\documentclass[utf8]{ctexart}

\usepackage[a4paper,left=1.25in,right=1.25in,top=1in,bottom=1in]{geometry}
\usepackage{listings}
\usepackage{graphicx}
\usepackage{caption}
\usepackage{subfigure}
\usepackage{booktabs}
\usepackage{amsmath}
\usepackage{amsthm}
\usepackage{amsfonts}
\usepackage{float}
\usepackage{indentfirst}
\usepackage{tikz}
\usetikzlibrary{shapes,arrows}
\usetikzlibrary{shapes.geometric, arrows}
\usepackage{algorithm}
\usepackage{algorithmic}
\usepackage{newclude}
\usepackage[perpage]{footmisc}

\graphicspath{ {images/} }
\raggedbottom	% 令页面在垂直方向向顶部对齐
\renewcommand\qedsymbol{QED}
\newcommand{\sign}[1]{\mathrm{sgn}(#1)}
\everymath{\displaystyle}   % 行内公式采用行间公式格式排列
\pagestyle{plain}

\title{《计算机辅助几何设计》第十一次作业}
\author{姓名:殷文良\qquad 学号:12435063}
\date{\today}

\begin{document}
\maketitle
\ctexset { section = { format={\Large \bfseries } } }

\section*{1. 思考题}
\subsection*{1.}
\begin{proof}
    双线性 Coons 曲面不一定位于其边界曲线的凸包内。\\
    构造一个简单的双线性 Coons 曲面:
    \begin{enumerate}
        \item \textbf{角点坐标:}\\
        $P(0,0) = (0,0,0), P(1, 0) = (1, 0, 0), P(0, 1) = (0, 1, 0), P(1, 1, 1) = (1, 1, -1)$;
        \item \textbf{边界曲线:}\\
        $P(0, v) = (0, v, 0)$,连接$P(0, 0)$和$P(0, 1)$;\\
        $P(1, v) = (1, v, -v)$,连接$P(1, 0)$和$P(1, 1)$;\\
        $P(u, 0) = (u, 0, 0)$,连接$P(0, 0)$和$P(1, 0)$;\\
        $P(u, 1) = (u, 1, -u)$,连接$P(0, 1)$和$P(1, 1)$;
    \end{enumerate}
    角点$P(0,0),P(1,0),P(0,1),P(1,1)$形成一个带有凹陷的四边形区域,这种设置会生成一个双线性 Coons 曲面,
    其中心区域由于插值的影响向下凹陷,导致靠近$P(1,1)$的曲面点落在边界曲线凸包的外部。
\end{proof}

\subsection*{2.}
\begin{proof}
    设曲面$P(u,v)$处处扭矢为0,则有$\frac{\partial^2 P(u,v)}{\partial u\partial v}\equiv 0$,对等式两边积分可得
    $$
    \frac{\partial P(u,v)}{\partial v} = C_0'(v).
    $$
    对上式两边进一步积分,有
    $$
    P(u,v) = C_1(u) + C_0(v) + C,
    $$
    其中,$\frac{d}{dv}C_0(v) = C_0'(v)$。不妨设$P(u,0) = C_1(u)$,从而有
    $$
    P(u,v) = C_1(u) + C_0(v) - C_0(0),
    $$
    即这是将曲线$C_1(u)$沿曲线$C_0(V)$平移得到的曲面。
\end{proof}

\subsection*{3.}
\begin{proof}
    根据双线性Coons曲面的定义,有
    $$
    \begin{aligned}
    S(u,v) &= Q_1(u,v) + Q_2(u, v) - \Delta(u,v)\\
    &= [P(0,v)(1-u) + P(1,v)u] + [P(u,0)(1-v)+P(u,1)v] \\
    &\quad - [P(0,0)(1-u)(1-v)+P(1,0)u(1-v)+P(0,1)v(1-u)+P(1,1)uv],
    \end{aligned}
    $$
    其中,$P(0,v),P(1,v),P(u,0),P(u,1)$均为三次曲线。\par
    上式第一项是关于$v$的三次多项式,关于$u$的线性多项式;
    第二项是关于$u$的三次多项式,关于$v$的线性多项式;第三项是关于$u,v$的双线性多项式。因此,
    $S(u,v)$是关于$u,v$的双三次多项式,即$S(u,v)$是双三次曲面。
\end{proof}


\end{document}
