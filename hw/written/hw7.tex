\documentclass[utf8]{ctexart}

\usepackage[a4paper,left=1.25in,right=1.25in,top=1in,bottom=1in]{geometry}
\usepackage{listings}
\usepackage{graphicx}
\usepackage{caption}
\usepackage{subfigure}
\usepackage{booktabs}
\usepackage{amsmath}
\usepackage{amsthm}
\usepackage{amsfonts}
\usepackage{float}
\usepackage{indentfirst}
\usepackage{tikz}
\usetikzlibrary{shapes,arrows}
\usetikzlibrary{shapes.geometric, arrows}
\usepackage{algorithm}
\usepackage{algorithmic}
\usepackage{newclude}
\usepackage[perpage]{footmisc}

\graphicspath{ {images/} }
\raggedbottom	% 令页面在垂直方向向顶部对齐
\renewcommand\qedsymbol{QED}
\newcommand{\sign}[1]{\mathrm{sgn}(#1)}
\everymath{\displaystyle}   % 行内公式采用行间公式格式排列
\pagestyle{plain}

\title{《计算机辅助几何设计》第七次作业}
\author{姓名:殷文良\qquad 学号:12435063}
\date{\today}

\begin{document}
\maketitle
\ctexset { section = { format={\Large \bfseries } } }

\section*{思考题 1}
\subsection*{1.}
\begin{proof}
    前者成立,后者不成立。
    \begin{itemize}
        \item 根据曲率和挠率的定义,曲率和挠率连续是$G^3$连续的必要条件。
        即如果两条曲线在连接点处$G^3$连续,那么其曲率和挠率必须是连续的。
        \item 但如果仅有曲率和挠率连续,可能仍存在其他因素导致$G^3$连续不成立。
        例如,如果两条曲线在连接点的切向或法向变化不匹配,即便它们的曲率和挠率连续,也可能导致$G^3$不连续。
    \end{itemize}
\end{proof}

\subsection*{2.}
\begin{proof}
    \item 第二类边界条件:$P'(a)=m_0,P'(b)=m_n$。
    \begin{equation}
        \begin{bmatrix}
            2+\bar{\nu}_1 & \mu_1 & 0 & \cdots & 0\\
            \lambda_2 & 2+\bar{\nu}_2 & \mu_2 & \cdots & 0\\
            \vdots & \ddots & \ddots & \ddots & \vdots\\
            0 & \cdots & \lambda_{n-2} & 2+\bar{\nu}_{n-2} & \mu_{n-2}\\
            0 & \cdots & 0 & \lambda_{n-1} & 2+\nu_{n-1}
        \end{bmatrix}
        \begin{bmatrix}
            m_1\\
            m_2\\
            \vdots\\
            m_{n-2}\\
            m_{n-1}
        \end{bmatrix} = 
        \begin{bmatrix}
            3D_1 - \lambda_1m_0\\
            3D_2\\
            \vdots\\
            3D_{n-2}\\
            3D_{n-1}-\mu_{n-1}m_n
        \end{bmatrix}.
    \end{equation}
    其中,$\bar{\nu}_i=\frac{\nu_i(u_i-u_{i-1})(u_{i+1}-u_i)}{2(u_{i+1}-u_{i-1})}$,
    $D_i=\lambda_i\frac{P_i-P_{i-1}}{u_i-u_{i-1}} + \mu_i\frac{P_{i+1}-P_i}{u_{i+1}-u_i}$
    \item 第三类边界条件:$P(a)=P(b),P'(a)=P'(b),P^"(a+)-P^"(b-)=(\nu_0+\nu_n)P'(a)$。
    \begin{equation}
        \begin{bmatrix}
            \bar{\nu}_0 & \Delta u_n & 0 & 0 & \cdots & \Delta u_1\\
           \lambda_1 &  2+\bar{\nu}_1 & \mu_1 & 0 & \cdots & 0\\
            0 &\lambda_2 & 2+\bar{\nu}_2 & \mu_2 & \cdots & 0\\
            0 & 0 & \vdots & \ddots & \ddots  & \vdots\\
            0 & 0 & \cdots & \lambda_{n-2} & 2+\bar{\nu}_{n-2} & \mu_{n-2}\\
            \mu_{n-1} & 0 & \cdots & 0 & \lambda_{n-1} & 2+\nu_{n-1}
        \end{bmatrix}
        \begin{bmatrix}
            m_0\\
            m_1\\
            m_2\\
            \vdots\\
            m_{n-2}\\
            m_{n-1}
        \end{bmatrix} = 
        3
        \begin{bmatrix}
            \Delta u_n\frac{P_1-P_0}{\Delta u_1} + \Delta u_1\frac{P_n-P_{n-1}}{\Delta u_{n}}\\
            D_1\\
            D_2\\
            \vdots\\
            D_{n-2}\\
            D_{n-1}
        \end{bmatrix}.
    \end{equation}
    其中,$\bar{\nu}_0=\frac{\Delta u_n\Delta u_1(\nu_0 + \nu_n) + 4(\Delta u_n + \Delta u_1)}{2}, m_0=m_n$。
\end{proof}

\end{document}
