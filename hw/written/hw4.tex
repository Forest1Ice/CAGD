\documentclass[utf8]{ctexart}

\usepackage[a4paper,left=1.25in,right=1.25in,top=1in,bottom=1in]{geometry}
\usepackage{listings}
\usepackage{graphicx}
\usepackage{subfigure}
\usepackage{booktabs}
\usepackage{amsmath}
\usepackage{amsthm}
\usepackage{amsfonts}
\usepackage{float}
\usepackage{indentfirst}
\usepackage{tikz}
\usetikzlibrary{shapes,arrows}
\usetikzlibrary{shapes.geometric, arrows}
\usepackage{algorithm}
\usepackage{algorithmic}
\usepackage{newclude}
\usepackage[perpage]{footmisc}

\graphicspath{ {images/} }
%\raggedbottom	% 令页面在垂直方向向顶部对齐
\renewcommand\qedsymbol{QED}
\newcommand{\sign}[1]{\mathrm{sgn}(#1)}
\everymath{\displaystyle}   % 行内公式采用行间公式格式排列
\pagestyle{plain}

\title{《计算机辅助几何设计》第四次作业}
\author{姓名:殷文良\qquad 学号:12435063}
\date{\today}

\begin{document}
\maketitle
\ctexset { section = { format={\Large \bfseries } } }

\section*{思考题1}
\subsection*{1.}
\begin{itemize}
    \item a.
    \begin{proof}
        由于$C_n^i = C_{n-1}^i + C_{n-1}^{i-1}$,因此
        \begin{equation*}
            M_{i,n}(t) = C_n^it^i
            = (C_{n-1}^i + C_{n-1}^{i-1})t^i
            = M_{i,n-1}(t) + tM_{i-1,n-1}(t).
        \end{equation*}
    \end{proof}
    \item b.
    \begin{algorithm}
        \caption{类 de Casteljau算法}
        \label{alg1}
        \renewcommand{\algorithmicrequire}{\textbf{Input:}}
        \renewcommand{\algorithmicensure}{\textbf{Output:}}
        \begin{algorithmic}[1]
            \REQUIRE $c_i,i=0,1,\dots,n,\quad t\in[0,1]$
            \ENSURE $c_0^n$

            \FOR{$i=0,1,\dots,n$}
            \STATE $c_i^0=c_i$
            \ENDFOR
            \FOR{$k=1,2,\dots,n$}
            \FOR{$i=0,1,\dots,n-k$}
            \STATE $c_i^k = c_{i}^{k-1} + tc_{i+1}^{k-1}$
            \ENDFOR
            \ENDFOR

            \RETURN $c_0^n$
        \end{algorithmic}
    \end{algorithm}
\end{itemize}

\subsection*{2.}
\begin{proof}
    令
    \begin{equation*}
        B_{i,n}'(t) = C_n^i(1-t)^{n-i-1}t^{i-1}(i-nt) = 0,
    \end{equation*}
    可得$t=\frac{i}{n}$,因此$B_{i,n}(t)$在$t=\frac{i}{n}$处达到极值,并且该极值是最大值。
\end{proof}

\subsection*{3.}
\begin{proof}
    设随机变量$\xi$服从二项分布,即
$$
\xi\sim B(n,t).
$$
则其分布列为
$$
P(\xi = i) = B_{i,n}(t),\quad i = 0,1,\dots,n.
$$
于是,$\xi$的期望为
$$
E\xi = \sum_{i=0}^niB_{i,n}(t).
$$
根据二项分布对参数$n$的可加性,以及两点分布$\eta$的期望为$t$,可知
$$
E\xi = nE\eta = nt.
$$
因此,我们有
$$
\sum_{i=0}^niB_{i,n}(t) = nt.
$$
\end{proof}

\section*{思考题2}


\end{document}
