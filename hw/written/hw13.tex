\documentclass[utf8]{ctexart}

\usepackage[a4paper,left=1.25in,right=1.25in,top=1in,bottom=1in]{geometry}
\usepackage{listings}
\usepackage{graphicx}
\usepackage{caption}
\usepackage{subfigure}
\usepackage{booktabs}
\usepackage{amsmath}
\usepackage{amsthm}
\usepackage{amsfonts}
\usepackage{float}
\usepackage{indentfirst}
\usepackage{tikz}
\usetikzlibrary{shapes,arrows}
\usetikzlibrary{shapes.geometric, arrows}
\usepackage{algorithm}
\usepackage{algorithmic}
\usepackage{newclude}
\usepackage[perpage]{footmisc}

\graphicspath{ {images/} }
\raggedbottom	% 令页面在垂直方向向顶部对齐
\renewcommand\qedsymbol{QED}
\newcommand{\sign}[1]{\mathrm{sgn}(#1)}
\everymath{\displaystyle}   % 行内公式采用行间公式格式排列
\pagestyle{plain}

\title{《计算机辅助几何设计》第十三次作业}
\author{姓名:殷文良\qquad 学号:12435063}
\date{\today}

\begin{document}
\maketitle
\ctexset { section = { format={\Large \bfseries } } }

\section*{1. 思考题}
\begin{proof}
    \begin{itemize}
        \item \textbf{均匀二次B样条}\\
        根据二次B样条细分算法,有
        $$
        \begin{aligned}
            P_{2i}^{j+1} &= \frac{3}{4}P_i^j+\frac{1}{4}P_{i+1}^j\\
            P_{2i+1}^{j+1} &= \frac{1}{4}P_{i}^j + \frac{3}{4}P_{i+1}^j.
        \end{aligned}
        $$
        对上式取差分,我们有
        $$
        \begin{aligned}
            (\Delta P^{j+1})_{2i} &= P_{2i+1}^{j+1} - P_{2i}^{j+1} = \frac{1}{4}(-2P_{i}^j+2P_{i+1}^{j})\\
            &= \frac{1}{2}(\Delta P^j)_i.
        \end{aligned}
        $$
        类似地,我们有$(\Delta P^{j+1})_{2i+1} = \frac{1}{2}(\Delta P^j)_{i+1}$。\par
        因此,$\|D\| = \frac{1}{2}$,根据细分收敛定理,均匀二次B样条细分曲线收敛。
        \item \textbf{均匀三次B样条}\\
        根据三次B样条细分算法,有
        $$
        \begin{aligned}
            P_{2i}^{j+1} &= \frac{1}{8}(4P_i^j+4P_{i+1}^j)\\
            P_{2i+1}^{j+1} &= \frac{1}{8}(P_{i-1}^j + 6P_i^j + P_{i+1}^j).
        \end{aligned}
        $$
        对上式取差分,我们有
        $$
        \begin{aligned}
            (\Delta P^{j+1})_{2i} &= P_{2i+1}^{j+1} - P_{2i}^{j+1} = \frac{1}{8}(P_{i-1}^j+2P_{i}^{j}-3P_{i+1}^{j})\\
            &= \frac{1}{8}(3(P_{i+1}^j-P_i^j) + (P_{i}^j-P_{i-1}^j)) = \frac{1}{8}(3(\Delta P^j)_{i} + (\Delta P^j)_{i-1}).
        \end{aligned}
        $$
        类似地,我们有$(\Delta P^{j+1})_{2i+1} = \frac{1}{8}(3(\Delta P^j)_{i} + (\Delta P^j)_{i+1})$。\par
        因此,$\|D\| = \frac{1}{2}$,根据细分收敛定理,均匀三次B样条细分曲线收敛。
    \end{itemize}
\end{proof}

\end{document}
